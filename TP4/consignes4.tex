\vspace{-10mm}
\section*{But de la manipulation}
\vspace{-6mm}
L’objectif de cette manipulation est de vous faire découvrir les fondements de l’électronique numérique et en particulier:
\vspace{3mm}
\begin{itemize}
	\item comprendre la représentation des données sous forme numérique
	\item réaliser quelques fonctions élémentaires au moyen de portes logiques
\end{itemize}

\vspace{3mm}
Plusieurs notions importantes du domaine de l’électronique numérique seront abordées, notamment: les codages binaire, hexadécimal et ASCII, les bus, les mémoires, les portes logiques, les tables de vérité, etc.

\section*{Pré-requis}
\vspace{-6mm}
La matière couverte par ce laboratoire est relative aux slides du cours 06 (Slides08: intro au numérique, Slides09: logique combinatoire asynchrone, Slides10: mémoires).\\
Une des manières d'assimiler ces concepts est précisément de lire les pages qui suivent (avant le labo!) car elles contiennent beaucoup de rappels théoriques.
\vspace{-5mm}

\section*{Prédéterminations}
\vspace{-5mm}
Il est recommandé d'essayer de répondre avant le labo aux exercices du point~\ref{Exercices1}.
\vspace{-5mm}

\section*{Objectifs}
\vspace{-6mm}
A la fin de ce laboratoire, vous devez être capable de comprendre et de réaliser :
\vspace{3mm}
\begin{itemize}
\item représentation des nombres en binaire
\item représentation des nombres en hexadécimal
\item représentation des caractères en code ASCII
\item dimensionnement d'une mémoire
\item opérations logiques combinatoires (y compris réalisation physique)
\item mémorisation (y compris réalisation physique)
\end{itemize}

\vspace{-3mm}
\section*{Remarque}
\vspace{-6mm}
Les exercices suivis de l’indication PRIOR sont à faire en priorité. Les autres sont à réaliser chez soi.

