\section*{Pré-requis :}
%\section{Pr?-requis}
%Avant la s?ance, vous aurez lu attentivement l??nonc? de la manipulation. Vous aurez par ailleurs relu les chapitres et sections suivants:
%\begin{itemize}
%	\item Chapitre 1 - Circuits ? ?l?ments concentr?s
%	\begin{itemize}
%		\item Section 1.6 - Puissance instantan?e, conventions et passivit?
%	\end{itemize}
%	\item Chapitre 5 - R?soudre un circuit : proc?dure de base et acc?l?rateur
%	\begin{itemize}
%		\item Section 5.1 -  Vocabulaire li? aux circuits
%		\begin{itemize}
%			\item 5.1.1 Rappels 
%			\item 5.1.2 Connexions s?rie et parall?le
%			\item 5.1.3 Branche
%			\item 5.1.4 Maille 
%		\end{itemize}
%		\item Section 5.2 - Lois de Kirchhoff
%	\end{itemize}
%\end{itemize}
%
%\vspace{5pt}
Avant la séance, vous aurez lu attentivement cet énoncé. Vous aurez par ailleurs relu les slides des 2 premiers cours, ainsi que les chapitres et sections suivants :
\begin{enumerate}
\item Chapitre 1 - Circuits à éléments concentrés
	\begin{enumerate}
    \item Circuit, éléments et noeuds
    \item Modèle de Kirchhoff
    \item Sens de lecture, charge et source
    \item Courant
    \item Tension(s)
    \end{enumerate}

\item Chapitre 5 - Résoudre un circuit : procédure de base et accélérateur
	\begin{enumerate}
    \item Vocabulaire lié aux circuits
	\item Lois de Kirchhoff
    \item Procédure canonique en 6 étapes
    \item Illustration: diviseur résistif
    \item Équivalences série et parallèle
    \end{enumerate}
\end{enumerate}


