\section*{Pré-requis}
Avant la séance, vous aurez lu attentivement cet énoncé. Vous aurez par ailleurs revu:\\
\begin{itemize}
\item Les notions d’électricité suivantes sont à relire avant la séance:
	\begin{itemize}
    \item adaptation d’impédance : §§4.1 à 4.3 du syllabus ELECH2001
	\item décibels et logarithmes: §9.1
	\item filtres RC et RL du premier ordre: §§6.2.4 et 6.3.3, 9.4 et 9.5, 9.6.1 et 9.6.2, 9.7.1 et 9.7.2
	\end{itemize}

\item Les notions d’électronique suivantes sont à relire avant la séance:
	\begin{itemize}
    \item amplification d’un signal analogique au moyen d’un amplifiateur opérationnel: fichier Slides06a
	\item montages inverseur et non-inverseur: fichier Slides06b
	\item relation entre gain et bande passante d’un ampli-op: fichier Slides06e
    \end{itemize}
\end{itemize}

\section*{But de la séance}
Les buts de cette séance sont :\\
\begin{itemize}
	\item Comprendre le principe de fonctionnement d'un filtre électronique et le dimensionner.
	\item Analyser un diagramme de Bode.
	\item Construire un dagramme de Bode.
	\item Pouvoir démontrer le gain d'un montage inverseur ou non-inverseur à base d'amplificateur opérationnel.
	\item Pouvoir dimensionner une chaîne d'amplification sur base d'un cahier des charges.
\end{itemize}
