\vspace{-5mm}
\section*{Pré-requis}
\vspace{-5mm}
Avant la séance, vous aurez lu attentivement cet énoncé et aurez refait les exercices de la séance précédente (TP2: Réalisation d'un amplificateur audio - 1\up{ère} partie : analyse du montage).
Vous aurez par ailleurs revu les notions suivantes (idem que pour le TP2):\\
\begin{itemize}
\item Les notions d’électricité suivantes sont à relire avant la séance:
	\begin{itemize}
    \item adaptation d’impédance : §§4.1 à 4.3 du syllabus ELECH2001
	\item décibels et logarithmes: §9.1
	\item filtres RC et RL du premier ordre: §§6.2.4 et 6.3.3, 9.4 et 9.5, 9.6.1 et 9.6.2, 9.7.1 et 9.7.2
	\end{itemize}

\item Les notions d’électronique suivantes sont à relire avant la séance:
	\begin{itemize}
    \item amplification d’un signal analogique au moyen d’un amplifiateur opérationnel: fichier Slides06a
	\item montages inverseur et non-inverseur: fichier Slides06b
	\item relation entre gain et bande passante d’un ampli-op: fichier Slides06e
    \end{itemize}
\end{itemize}

\vspace{-5mm}
\section*{But de la manipulation}
\vspace{-5mm}
Les buts de cette manipulation sont :\\
\begin{itemize}
	\item analyser un montage électronique courant (dans ce cas-ci: un ampli audio);
	\item illustrer quelques fonctions de base en électronique ;
	\item illustrer l'utilisation des amplificateurs opérationnels dans une application réaliste;
    \item Câbler, dépanner et faire fonctionner un montage électronique
\end{itemize}

\vspace{-5mm}
\section*{Aquis d'apprentissage}
\vspace{-5mm}
A la fin de ce laboratoire, vous devez :\\
\begin{itemize}
\item être capable d'expliquer le fonctionnement de notre ampli audio;
\item avoir réalisé qu'on peut comprendre le fonctionnement d'un circuit électronique complexe en identifiant des blocs (étages ampli-op, filtres) et en les analysant séparément, pour après comprendre le fonctionnement de l'ensemble;
\item utiliser ce principe pour analyser un montage électronique.
\item être capable de câbler proprement un circuit électronique complexe sur le protoboard.
\end{itemize}