\section*{Pré-requis:}
Avant la séance, vous aurez lu attentivement cet énoncé. Vous aurez par ailleurs relu les slides des 2 premiers cours, ainsi que les chapitres et sections suivants :

\begin{enumerate}
	\item Chapitre 5 - Résoudre un circuit : procédure de base et accélérateur
	\begin{enumerate}
    \item Vocabulaire lié aux circuits
	\item Lois de Kirchhoff
    \item Procédure canonique en 6 étapes
    \item Illustration: diviseur résistif
    \item Équivalences série et parallèle
    \end{enumerate}
	\item Chapitre 7 - Résoudre un circuit réactif dans le domaine fréquentiel
	\begin{enumerate}
		\item Analyse fréquentielle: circuits linéaires en régime sinusoïdal
        \item Caractérisation des fonctions périodiques
        \item Phaseurs
		\item Impédances et admittances
	\end{enumerate}
\end{enumerate}